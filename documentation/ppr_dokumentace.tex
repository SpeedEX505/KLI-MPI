\documentclass[11pt]{article}
\usepackage{array, xcolor, lipsum, bibentry}
\usepackage[margin=2.5cm]{geometry}
\usepackage[czech]{babel}	%cesky jazyk	
\usepackage[utf8]{inputenc}
\usepackage[T1]{fontenc}
\usepackage[T5]{fontenc}	%kvuli memu jmenu 
\usepackage{graphicx}



 
\begin{document}

\begin{center}
	\bf Semestralní projekt MI-PAR 2014/2015:\\[5mm]
	 Paralelní algoritmus pro řešení problému maximální klika grafu\\[5mm]
       	Tomáš Šabata\\
	 Ph{\'u} H{\h a}i B{\` u}i\\[2mm]
	magisterské studijum, FIT ČVUT, Kolejní 550/2, 160 00 Praha 6\\[2mm]
	\today
\end{center}

\section{Definice problému a popis sekvenčního algoritmu}

Popište problém, který váš program řeší. Jako výchozí použijte text
zadání, který rozšiřte o přesné vymezení všech odchylek, které jste
vůči zadání během implementace provedli (např.  úpravy heuristické
funkce, organizace zásobníku, apod.). Zmiňte i případně i takové
prvky algoritmu, které v zadání nebyly specifikovány, ale které se
ukázaly jako důležité.  Dále popište vstupy a výstupy algoritmu
(formát vstupních a výstupních dat). Uveďte tabulku nameřených časů
sekvenčního algoritmu pro různě velká data.
\subsection{Vstupní data: }

%TODO - Tomas
\section{Popis paralelního algoritmu a jeho implementace v MPI}

Popište paralelní algoritmus, opět vyjděte ze zadání a přesně
vymezte odchylky, zvláště u algoritmu pro vyvažování zátěže, hledání
dárce, ci ukončení výpočtu.  Popište a vysvětlete strukturu
celkového paralelního algoritmu na úrovni procesů v MPI a strukturu
kódu jednotlivých procesů. Např. jak je naimplementována smyčka pro
činnost procesů v aktivním stavu i v stavu nečinnosti. Jaké jste
zvolili konstanty a parametry pro škálování algoritmu. Struktura a
sémantika příkazové řádky pro spouštění programu.



\section{Naměřené výsledky a vyhodnocení}

\begin{enumerate}
\item Zvolte tři instance problému s takovou velikostí vstupních dat, pro které má
sekvenční algoritmus časovou složitost kolem 5, 10 a 15 minut. Pro
meření čas potřebný na čtení dat z disku a uložení na disk
neuvažujte a zakomentujte ladící tisky, logy, zprávy a výstupy.
\begin{table}[h]
	\caption{Naměřené hodnoty}
	\label{tab:namereneHodnoty}
	\centering
	\begin{tabular}{| c || c | c | c |}
		\hline
		\textbf{Počer jader $\backslash$ Graf} & \textbf{55 uzlový} & \textbf{65 uzlový} & \textbf{70 uzlový} \\
		\hline \hline
		1 & 281 s & 615 s & 901 s  \\
		\hline
		2 & 214 s & 797 s & 925 s \\
		\hline
		4 & 216 s & 727 s & 704 s  \\
		\hline
		8 & 177 s & 328 s & 326 s \\
		\hline
		16 & 171 s & 90 s & 132 s  \\
		\hline
		32 & 178 s & 95 s & 134 s  \\
		\hline
	\end{tabular}
\end{table}
\item Měřte paralelní čas při použití $i=2,\cdot,32$ procesorů na sítích Ethernet a InfiniBand.
%\item Pri mereni kazde instance problemu na dany pocet procesoru spoctete pro vas algoritmus dynamicke delby prace celkovy pocet odeslanych zadosti o praci, prumer na 1 procesor a jejich uspesnost.
%\item Mereni pro dany pocet procesoru a instanci problemu provedte 3x a pouzijte prumerne hodnoty.
\item Z naměřených dat sestavte grafy zrychlení $S(n,p)$. Zjistěte, zda a za jakych podmínek
došlo k superlineárnímu zrychlení a pokuste se je zdůvodnit.
\item Vyhodnoďte komunikační složitost dynamického vyvažování zátěže a posuďte
vhodnost vámi implementovaného algoritmu pro hledání dárce a dělení
zásobníku pri řešení vašeho problému. Posuďte efektivnost a
škálovatelnost algoritmu. Popište nedostatky vaší implementace a
navrhněte zlepšení.
\item Empiricky stanovte
granularitu vaší implementace, tj., stupeň paralelismu pro danou
velikost řešeného problému. Stanovte kritéria pro stanovení mezí, za
kterými již není učinné rozkládat výpočet na menší procesy, protože
by komunikační náklady prevážily urychlení paralelním výpočtem.

\end{enumerate}

%%%%%%%%%%%%%%%%%%%%%%%%%%%%%%%%%%%%%%%%%%%%%%%%%%%%%%%%%%%%%%%%5
\section{Závěr}

Celkové zhodnocení semestrální práce a zkušenosti získaných během
semestru.

\section{Literatura}

\appendix

\section{Návod pro vkládání grafů a obrázků do Latexu}

Nejjednodušší způsob vytvoření obrázku je použít vektorový grafický
editor (např. xfig nebo jfig), ze kterého lze exportovat buď
\begin{itemize}
\item postscript formáty (ps nebo eps formát) nebo
\item latex formáty (v pořadí prostý latex, latex s macry epic, eepic, eepicemu). Uvedené pořadí odpovídá růstu
komplikovanosti obrázků který formát podporuje (prostá latex macra
umožnují pouze jednoduché, epic makra něco mezi, je třeba
vyzkoušet).

\end{itemize}
Následující příklady platí pro všechny případy.

Obrázek v postscriptu, vycentrovaný a na celou šířku stránky, s
popisem a číslem. Všimnete si, jak řídit velikost obrazku.
%\begin{figure}[ht]
%\epsfysize=3cm \centerline{\epsfbox{VasObrazek.ps}} \caption{Popis
%vašeho obrazku} \label{labelvasehoobrazku}
%\end{figure}

Obrázek pouze vložený mezi řádky textu, bez popisu a číslování.\\
%\epsfxsize=1cm
%\rule{0pt}{0pt}\hfill\epsfbox{VasObrazek.ps}\hfill\rule{0pt}{0pt}

%Latexovské obrázky maji přípony *.latex, *.epic, *.eepic, a
%*.eepicemu, respective.
%\begin{figure}[ht]
%\begin{center}
%\input VasObrazek.latex
%\end{center}
%\caption{Popis vašeho obrázku} \label{l1}
%\end{figure}
%Vypuštením závorek {\tt figure} dostanete opět pouze rámeček v textu
%bez čísla a popisu.

%Takhle jednoduše můžete poskládat obrázky vedle sebe.
%\begin{center}
%\setlength{\unitlength}{0.1mm}\input VasObrazek.epic
%\hglue 5mm
%\setlength{\unitlength}{0.15mm}\input VasObrazek.eepic
%\hglue 5mm
%\setlength{\unitlength}{0.2mm}\input VasObrazek.eepicemu
%\end{center}
%Řídit velikost latexovskych obrázků lze příkazem
%\begin{verbatim}
%\setlength{\unitlength}{0.1mm}
%\end{verbatim}
%které mění měřítko rastru obrázku, Tyto příkazy je ale současně
%nutné vyhodit ze souboru, který xfig vygeneroval.

%Pro vytváření grafu lze použít program gnuplot, který umí generovat
%postscriptovy soubor, ktery vložíte do Latexu výše uvedeným
%způsobem.
\end{document}